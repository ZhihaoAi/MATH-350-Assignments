\documentclass[10pt]{report}

\usepackage{verbatim}
%\usepackage{subcaption} % for subfigures
%\usepackage{amsthm} % for QED
%\usepackage{algpseudocode} % for pseudo-code
\usepackage{mathtools} % for \xRightarrow?

\usepackage{listings} % for code
\lstset
{
	language=Matlab,
	frame=single,
	basicstyle=\footnotesize,
	numbers=left,
	stepnumber=1,
	showstringspaces=false,
	tabsize=4,
	breaklines=true,
	breakatwhitespace=false,
}

\usepackage{siunitx} % for scientific notation
% for `e' in scientific notation
\sisetup{output-exponent-marker=\ensuremath{\mathrm{e}}}

\usepackage{float} % for figure [H]
\usepackage{booktabs} % for tabular
\usepackage{caption} % for \caption*
\usepackage[export]{adjustbox} % for valign=t
\usepackage{array} % for column type m
\usepackage{verbatim}
\usepackage{graphicx}
\graphicspath{ {imgs/} }
\usepackage{fancyhdr}
\usepackage{amssymb}
\usepackage{amsmath}

%%%%%% Pagination
\setlength{\topmargin}{-.3 in}
\setlength{\oddsidemargin}{0in}
\setlength{\evensidemargin}{0in}
\setlength{\textheight}{9.in}
\setlength{\textwidth}{6.5in}

%Title page
\newcommand{\hwTitle}{Homework \#3}
\newcommand{\hwCourse}{Introduction to Computational Mathematics}
\newcommand{\hmwkClassInstructor}{Professor Shuwang Li}

\title{
	\vspace{2in}
	\textmd{\textbf{\hwCourse\\\hwTitle}}\\
	\vspace{0.3in}\large{\textit{\hmwkClassInstructor}}
	\vspace{3in}
}

%\title{Homework 1}
\author{\textbf{Zhihao Ai}}
\date{}

%Header setting.
\pagestyle{fancy}
\fancyhead[L]{Zhihao Ai}
\fancyhead[C]{Math 350}
\fancyhead[R]{Homework 3}
%%%%%%

%\displaystyle apply to all
\everymath{\displaystyle}

%Custom commands.
\newcommand{\ds}{\displaystyle}
\newcommand{\eva}[2] {\left. #1 \right|_{#2}}
\newcommand{\dintt}[4] {\int_{#1}^{#2} #3 d#4}

\newcolumntype{C}{ >{\centering\arraybackslash} m{3em} }
\newcolumntype{D}{ >{\centering\arraybackslash} m{4em} }
\newcolumntype{N}{ >$ c <$}

\newcommand{\abs}[1] {\left| #1 \right|}
\newcommand{\norm}[2][\infty] {\left\Vert \mathbf{#2} \right\Vert_#1}

\begin{document}

\maketitle

\section*{Part 1. Reading Assignment}

\section*{Part 2. Fundamental Concepts/Ideas}
\begin{enumerate}
	\item 
	The exact solution to the following linear system is $\mathbf{x} = (1, -2, 3)^\mathrm{T}$.
	\begin{align*}
		10x_1 - 2x_2 - x_3 &= 11\\
		-2x_1 + 10x_2 - x_3 &= -25\\
		-x_1 - 2x_2 + 5x_3 &= 18
	\end{align*}
	\begin{enumerate}
		\item 
		Solve the system using Gauss--Jacobi iterative method with $\mathbf{x}^{(0)} = (0, 0, 0)^\mathrm{T}$, stop the iteration if tolerance, $tol = \norm{x^{(\mathit k + \mathrm 1)} - x^{(\mathit{k})}} \le 0.5\times 10^{-2}$
		
		According to the coefficient matrix,
		\begin{align*}
			M &= -D^{-1}(A-D) = -\begin{pmatrix}
			1/10 & 0 & 0\\
			0 & 1/10 & 0\\
			0 & 0 & 1/5
			\end{pmatrix}
			\begin{pmatrix}
			0 & -2 & -1\\
			-2 & 0 & -1\\
			-1 & -2 & 0
			\end{pmatrix}
			=
			\begin{pmatrix}
			0 & 1/5 & 1/10\\
			1/5 & 0 & 1/10\\
			1/5 & 2/5 & 0
			\end{pmatrix}
			\\
			\mathbf{g} &= D^{-1}\mathbf{b} = \begin{pmatrix}
			1/10 & 0 & 0\\
			0 & 1/10 & 0\\
			0 & 0 & 1/5
			\end{pmatrix}
			\begin{pmatrix}
			11\\
			-25\\
			18
			\end{pmatrix}
			=
			\begin{pmatrix}
			11/10\\
			-5/2\\
			18/5
			\end{pmatrix}
		\end{align*}
		Apply $\mathbf{x}^{(k)} = M\mathbf{x}^{(k-1)} + \mathbf{g}$ iteratively with $\mathbf{x}^{(0)} = (0, 0, 0)^\mathrm{T}$ using Matlab:
		\lstinputlisting{hw3p1a.m}
		The code above produces:
		\begin{table}[H]
			\centering
			\begin{tabular}{NNN} \toprule
				k & x^{(k)} & tol \\ \midrule
				1 & (1.1000,-2.5000,3.6000)^\mathrm{T} & 3.6000\\
				2 & (0.9600,-1.9200,2.8200)^\mathrm{T} & 0.7800\\
				3 & (0.9980,-2.0260,3.0240)^\mathrm{T} & 0.2040\\
				4 & (0.9972,-1.9980,2.9892)^\mathrm{T} & 0.0348\\
				5 & (0.9993,-2.0016,3.0002)^\mathrm{T} & 0.0110\\
				6 & (0.9997,-2.0001,2.9992)^\mathrm{T} & 0.0015\\
				\bottomrule
			\end{tabular}
		\end{table}
		
		\item 
		Solve the system using Gauss--Seidel iterative method with $\mathbf{x}^{(0)} = (0, 0, 0)^\mathrm{T}$, stop the iteration if tolerance, $tol = \norm{x^{(\mathit k + \mathrm 1)} - x^{(\mathit{k})}} \le 0.5\times 10^{-2}$
		
		\item 
		Interchange equation \#2 and \#3, do (a) and (b).
		
		\item 
		What can you learn from this problem?
	\end{enumerate}

	\item 
	Given $\sin(0.32)=0.314567, \sin(0.34)=0.333487, \sin(0.36)=0.352274$. Compute $\sin(0.3367)$ by Lagrange interpolation polynomial.
	\begin{enumerate}
		\item 
		A linear interpolation using the first two points.
		
		\item 
		A quadratic interpolation using all three points.
		
		\item 
		Estimate the absolute error in part (a) and (b).
	\end{enumerate}
\end{enumerate}

\section*{Part 3. Computer Assignments}
\begin{enumerate}
	\item 
	Problem 3.3(a)(b) @ Page 19
	\begin{enumerate}
		\item
		Interpolate these data by each of the four interpolants discussed in this
		chapter: \verb|piecelin|, \verb|polyinterp|, \verb|splinetx|, and \verb|pchiptx|. Plot the results for $-1 \le x \le 1$.
		\begin{table}[H]
			\centering
			\begin{tabular}{*{7}{N}} \toprule
				x & -1.00 & -0.96 & -0.65 & 0.10 & 0.40 & 1.00\\ \midrule
				y & -1.0000 & -0.1512 & 0.3860 & 0.4802 & 0.8838 & 1.0000\\
				\bottomrule
			\end{tabular}
		\end{table}
		
		\item 
		What are the values of each of the four interpolants at $x = -0.3$? Which
		of these values do you prefer? Why?
	\end{enumerate}
	
	\item 
	Problem 3.4 @ Page 20
	\begin{enumerate}
		\item 
		Interpolate both functions of $x$ and $y$ on a finer grid and plot the result. Do the same thing with \verb|pchiptx|. Which do you prefer?
		
		\item 
		Can you tell if Figure 3.11 was done with \verb|splinetx| or \verb|pchiptx|?
	\end{enumerate}
\end{enumerate}

\end{document}


