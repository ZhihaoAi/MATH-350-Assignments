\documentclass[10pt]{report}

\usepackage{verbatim}
%\usepackage{subcaption} % for subfigures
%\usepackage{amsthm} % for QED
%\usepackage{algpseudocode} % for pseudo-code
\usepackage{mathtools} % for \xRightarrow?

\usepackage{listings} % for code
\lstset
{
	language=Matlab,
	frame=single,
	basicstyle=\footnotesize,
	numbers=left,
	stepnumber=1,
	showstringspaces=false,
	tabsize=4,
	breaklines=true,
	breakatwhitespace=false,
}

\usepackage{siunitx} % for scientific notation
% for `e' in scientific notation
\sisetup{output-exponent-marker=\ensuremath{\mathrm{e}}}

\usepackage{float} % for figure [H]
\usepackage{booktabs} % for tabular
\usepackage{caption} % for \caption*
\usepackage[export]{adjustbox} % for valign=t
\usepackage{array} % for column type m
\usepackage{verbatim}
\usepackage{graphicx}
\graphicspath{ {imgs/} }
\usepackage{fancyhdr}
\usepackage{amssymb}
\usepackage{amsmath}

%%%%%% Pagination
\setlength{\topmargin}{-.3 in}
\setlength{\oddsidemargin}{0in}
\setlength{\evensidemargin}{0in}
\setlength{\textheight}{9.in}
\setlength{\textwidth}{6.5in}

%Title page
\newcommand{\hwTitle}{Homework \#2}
\newcommand{\hwCourse}{Introduction to Computational Mathematics}
\newcommand{\hmwkClassInstructor}{Professor Shuwang Li}

\title{
	\vspace{2in}
	\textmd{\textbf{\hwCourse\\\hwTitle}}\\
	\vspace{0.3in}\large{\textit{\hmwkClassInstructor}}
	\vspace{3in}
}

%\title{Homework 1}
\author{\textbf{Zhihao Ai}}
\date{}

%Header setting.
\pagestyle{fancy}
\fancyhead[L]{Zhihao Ai}
\fancyhead[C]{Math 350}
\fancyhead[R]{Homework 2}
%%%%%%

%\displaystyle apply to all
\everymath{\displaystyle}

%Custom commands.
\newcommand{\ds}{\displaystyle}
\newcommand{\eva}[2] {\left. #1 \right|_{#2}}
\newcommand{\dintt}[4] {\int_{#1}^{#2} #3 d#4}

\newcolumntype{C}{ >{\centering\arraybackslash} m{3em} }
\newcolumntype{D}{ >{\centering\arraybackslash} m{4em} }

\newcommand{\abs}[1] {\left| #1 \right|}
\newcommand{\norm}[2][\infty] {\left\Vert \mathbf{#2} \right\Vert_#1}

\begin{document}

\maketitle

\section*{Part 1. Reading Assignment}

\section*{Part 2. Fundamental concepts/ideas}
\begin{enumerate}
	\item 
	Consider the linear system
	\begin{align*}
	3x_1 + 4x_2 + 3x_3 &= 16 \\
	x_1 + 5x_2 - x_3 &= -12 \\
	6x_1 + 3x_2 + 7x_3 &= 102
	\end{align*}
	\begin{enumerate}
		\item 
		Find matrix $L$ and $U$ using four significant digits.
		\begin{align*}
		\begin{pmatrix}
		1 & 0 & 0\\
		-0.3333 & 1 & 0\\
		-2.000 & 0 & 1
		\end{pmatrix}
		\begin{pmatrix}
		3 & 4 & 3\\
		1 & 5 & -1\\
		6 & 3 & 7
		\end{pmatrix}
		&=
		\begin{pmatrix}
		3 & 4 & 3\\
		0 & 3.667 & -2.000\\
		0 & -5.000 & 1.000
		\end{pmatrix}
		\tag{$L_1 A= A_1$}
		\\
		\begin{pmatrix}
		1 & 0 & 0\\
		0 & 1 & 0\\
		0 & 1.364  & 1
		\end{pmatrix}
		\begin{pmatrix}
		3 & 4 & 3\\
		0 & 3.667 & -2.000\\
		0 & -5.000 & 1.000
		\end{pmatrix}
		&=
		\begin{pmatrix}
		3 & 4 & 3\\
		0 & 3.667 & -2.000\\
		0 & 0 & -1.728
		\end{pmatrix}
		\tag{$L_2 A_1= U$}
		\end{align*}
		Because $L = L_1^{-1} L_2^{-1}$, from the multipliers in $L_1$ and $L_2$ we obtain
		\[
		L = \begin{pmatrix}
		1 & 0 & 0\\
		0.3333 & 1 & 0\\
		2.000 & -1.364  & 1
		\end{pmatrix}
		,\ 
		U = \begin{pmatrix}
		3 & 4 & 3\\
		0 & 3.667 & -2.000\\
		0 & 0 & -1.728
		\end{pmatrix}
		\]
		
		\item 
		Solve the linear system using $L$ and $U$ found in part (a).
		
		Since $L\mathbf{y} = \mathbf{b}$, by forward substitution,
		\[
			\begin{pmatrix}
			1 & 0 & 0\\
			0.3333 & 1 & 0\\
			2.000 & -1.364  & 1
			\end{pmatrix}
			\mathbf{y}
			=
			\begin{pmatrix}
			16\\
			-12\\
			102
			\end{pmatrix}
			\Rightarrow
			\mathbf{y}
			=
			\begin{pmatrix}
			16\\
			-17.3328\\
			46.3581
			\end{pmatrix}
		\]
		Also $U\mathbf{x} = \mathbf{y}$, by back substitution,
		\[
		\begin{pmatrix}
		3 & 4 & 3\\
		0 & 3.667 & -2.000\\
		0 & 0 & -1.728
		\end{pmatrix}
		\mathbf{x}
		=
		\begin{pmatrix}
		16\\
		-17.3328\\
		46.3581
		\end{pmatrix}
		\Rightarrow
		\mathbf{x}
		\approx
		\begin{pmatrix}
		57.9724\\
		-19.3586\\
		-26.8276
		\end{pmatrix}
		\]
	\end{enumerate}
	\textit{Things learned}: $LU$ decomposition make it fast when we want to solve systems with the same $A$ but for different $\mathbf{b}$'s. $L$ can be found using Gaussian multipliers. With four significant digits, the difference between terms of $LU$ and those of $A$ is generally within $10^{-3}$. The exact solution $\mathbf{x} \approx (58, -19.3684, -26.8421)^{\rm T}$ is pretty close to what we calculated, which differs within $10^{-1}$.

	\item 
	Solve the following linear system and compare your solution with the exact solution $(2, 1)$.
	\begin{align*}
		x_1 + 2x_2 &= 4\\
		2x_1 + 3.999x_2 &= 7.999
	\end{align*}
	\begin{enumerate}
		\item 
		Suppose the right hand side $b = (4, 7.999)$ is slightly perturbed to be $b_1=(4.001, 7.998)$, solve the system using the new right hand side $b_1$.
		\[
		\left(
		\begin{array}{@{}cc|c@{}}
		1& 2& 4.001\\
		2& 3.999& 7.998
		\end{array}
		\right)
		\xRightarrow{\text{R.R.}}
		\left(
		\begin{array}{@{}cc|c@{}}
		1& 2& 4.001\\
		0& -0.001& -0.004
		\end{array}
		\right)
		\]
		yields $x_2 = 4$. Using back substition, $x_1 = (4.001-4*2)/1 = -3.999$. Error $\mathbf{e} = \mathbf{x} - \mathbf{x^*} = (5.999, -3)^{\rm T}$.
		
		\item 
		Solve the new system where the left hand side is slightly perturbed to
		\begin{align*}
		1.001x_1 + 2.001x_2 &= 4\\
		2.001x_1 + 3.998x_2 &= 7.999
		\end{align*}
		\[
		\left(
		\begin{array}{@{}cc|c@{}}
		1.001& 2.001& 4\\
		2.001& 3.998& 7.999
		\end{array}
		\right)
		\xRightarrow{\text{R.R.}}
		\left(
		\begin{array}{@{}cc|c@{}}
		1.001& 2.001& 4\\
		0& -0.002& 0.003
		\end{array}
		\right)
		\]
		yields $x_2 = 0.003/(-0.002) = -1.5$. Using back substition, $x_1 = (4-(-1.5)*2.001)/1.001 = 6.995$. Error $\mathbf{e} = \mathbf{x} - \mathbf{x^*} = (-4.995, 2.5)^{\rm T}$.
	\end{enumerate}
	\textit{Things learned:} The matrix is near singular because the second equation is almost twice the first one. Since $cond(A)$ is large, the matrix is ill-conditioned. For such matrix, small change in $A$ or $\mathbf{b}$ will result in huge change in $\mathbf{x}$.
	
	\item 
	Using the maximum norm,
	\begin{enumerate}
		\item 
		compute the ratio between the relative change in solution and relative change in right hand side, for problem 2(a).
		\[
		\left.\frac{\norm{\delta x}}{\norm{x}}\right/\frac{\norm{\delta b}}{\norm{b}}
		= \left.\frac{5.999}{2}\right/\frac{0.001}{7.999}
		\approx 23993
		\]
		
		\item 
		compute the ratio between the relative change in solution and relative change in the coefficient matrix, for problem 2(b).
		\[
		\left.\frac{\norm{\delta x}}{\norm{x}}\right/\frac{\norm{\delta A}}{\norm{A}}
		= \left.\frac{4.995}{2}\right/\frac{0.002}{5.999}
		\approx 7497
		\]
		
		\item 
		compute the condition number of the original coefficient matrix used in problem 2. Is this linear system well-conditioned?
		\[
		\norm{A}\norm{A^\mathrm{-1}} 
			= \norm{\begin{pmatrix}
			1 & 2\\ 2 & 3.999
			\end{pmatrix}} 
			\norm{\begin{pmatrix}
				-3999 & 2000\\ 2000 & -1000
				\end{pmatrix}} 
			= 5.999 * 5999 \approx 35980
		\]
		This system is ill-conditioned.
	\end{enumerate}
	\textit{Things learned:} When a matrix is near singular, the condition number of it becomes huge. For a matrix with a big condition number, small change in $\mathbf{A}$ or $\mathbf{b}$ would result in huge change in $\mathbf{x}$. Condition number of a matrix, $\text{cond}(\bf A) = \left\Vert \mathbf{A}\right\Vert \left\Vert \mathbf{A}^{\rm -1}\right\Vert$.
	
	\item 
	Compute the spectral radius, $\rho(A)$ of matrix $A = \begin{pmatrix}
	1 & 1 & 0\\
	1 & 2 & 1\\
	-1 & 1 & 2
	\end{pmatrix}$ and its $L_1$, $L_2$ and $L_\infty$ norm.
	\begin{align*}
	\text{det}(A-\lambda I) = 
	\begin{vmatrix}
	1-\lambda & 1 & 0\\
	1 & 2-\lambda & 1\\
	-1 & 1 & 2-\lambda
	\end{vmatrix} &= 0 \to \lambda_1 = 3, \lambda_2 = 2, \lambda_3 = 0 \\
	\rho(A) &= \max_i{\abs{\lambda_i}} = 3\\
	\norm[1]{A} &= \max_{j}{\norm[1]{A(:,\mathit{j})}} = 4\\
	\norm{A} &= \max_{i}{\norm[1]{A(\mathit{i},:)}} = 4\\
	\text{det}(A^TA-\lambda I) = 
	\begin{vmatrix}
	2-\lambda & 3 & 0\\
	3 & 6-\lambda & 3\\
	0 & 3 & 6-\lambda
	\end{vmatrix} &= 0
	\to
	\lambda_1 = 7+\sqrt{7}, \lambda_2 = 7-\sqrt{7}, \lambda_3 = 0 \\
	\norm[2]{A} &= \sqrt{\rho(A^TA)} = \sqrt{7+\sqrt{7}} \approx 3.106
	\end{align*}
	\textit{Things learned:} Spectral radius of a matrix equals the maximum eigenvalue of it. $L_1$ norm of a matrix is the max column sum. $L_\infty$ norm of a matrix is the max row sum. $L_\infty$ norm of a matrix equals $\sqrt{\rho(A^TA)}$.
\end{enumerate}

\section*{Part 3. Computer assignments}
\begin{enumerate}
	\item
	Problem 2.3 @Page 32\\
	Solve a system of equations to find the vector $f$ of member forces.
	\lstinputlisting{hw2p1.m}
	The script above outputs $f = [-20\sqrt{2}, 20, 10, -30, 10\sqrt{2}, 20, 0, -30, 5\sqrt{2}, 25, 20, -25\sqrt{2}, 25]^{\rm T}$.
	
	\item 
	Problem 2.5 @Page 35
	\begin{enumerate}
		\item 
		Which of the following families of matrices are positive definite?
		\begin{verbatim}
			M = magic(n)
			H = hilb(n)
			P = pascal(n)
			I = eye(n,n)
			R = randn(n,n)
			R = randn(n,n); A = R’ * R
			R = randn(n,n); A = R’ + R
			R = randn(n,n); I = eye(n,n); A = R’ + R + n*I
		\end{verbatim}
		Excluding $n = 1$ and applying \verb|[~, p] = chol(·)| to each family of matrices and check if $p = 0$, which means the matrix is positive definite, we can get the following conclusions:\\
		\verb|M = magic(n)| is not positive definite.\\
		\verb|H = hilb(n)|, \verb|P = pascal(n)|, \verb|I = eye(n,n)| and \verb|A = R’ * R| are positive definite.\\
		For \verb|R = randn(n,n)| and \verb|A = R’ + R|, they are generally not positive definite but sometimes they can be positive definite depending on what the randomly generated matrix is like.\\
		For \verb|A = R’ + R + n*I|, it is generally positive definite but sometimes they can be not positive definite depending on what the randomly generated matrix is like.
		
		\item 
		If the matrix $R$ is upper triangular, then equating individual elements in
		the equation $A = R^T R$ gives
		\[
		a_{kj} = \sum_{i=1}^{k} r_{ik}r_{ij},\ k\le j
		\]
		Using these equations in different orders yields different variants of the Cholesky
		algorithm for computing the elements of $R$. What is one such algorithm?
		\lstinputlisting{hw2p2b.m}
		This algorithm above yields one column at a time. For example, \verb|hw2p2b(pascal(4))| would outputs:
		\[
		\begin{pmatrix}
		1 & 1 & 1 & 1\\
		1 & 2 & 3 & 4\\
		1 & 3 & 6 & 10\\
		1 & 4 & 10 & 20
		\end{pmatrix}
		\xRightarrow{\text{hw2p2\_2(pascal(4))}}
		\begin{pmatrix}
		1 & 1 & 1 & 1\\
		0 & 1 & 2 & 3\\
		0 & 0 & 1 & 3\\
		0 & 0 & 0 & 1
		\end{pmatrix}
		\]
	\end{enumerate}

	\item 
	Problem 2.7 @ Page 37\\
	Modify the \verb|lutx| function so that it returns four outputs. Write a function \verb|mydet(A)| that uses your modified \verb|lutx| to compute the
	determinant of $A$. 
	\lstinputlisting{mydet.m}
	The change in \verb|lutx| is made at line 26, which updates the sign when a permutation happens.

	\item 
	Problem 2.19 @ Page 40\\
	For $n = 100$, solve this tridiagonal system of equations three different ways:
	\begin{align*}
	2x_1 - x_2 &= 1,\\
	-x_{j-1} + 2x_j - x_{j+1} &= j,\ j = 2, \dots, n-1,\\
	-x_{n-1} + 2x_n &= n.
	\end{align*}
	\begin{enumerate}
		\item 
		Use \verb|diag| three times to form the coefficient matrix and then use \verb|lutx| and \verb|bslashtx| to solve the system.
		\lstinputlisting[firstline=1,lastline=4]{hw2p4a.m}
		produces the result
		\begin{multline*}
			\mathbf{x} = \num{1.0e+04} * [0.1700,0.3399,0.5096,0.6790,0.8480,1.0165,1.1844,1.3516,1.5180,1.6835,1.8480,\\
			2.0114,2.1736,2.3345,2.4940,2.6520,2.8084,2.9631,3.1160,3.2670,3.4160,3.5629,3.7076,3.8500,\\
			3.9900,4.1275,4.2624,4.3946,4.5240,4.6505,4.7740,4.8944,5.0116,5.1255,5.2360,5.3430,5.4464,\\
			5.5461,5.6420,5.7340,5.8220,5.9059,5.9856,6.0610,6.1320,6.1985,6.2604,6.3176,6.3700,6.4175,\\
			6.4600,6.4974,6.5296,6.5565,6.5780,6.5940,6.6044,6.6091,6.6080,6.6010,6.5880,6.5689,6.5436,\\
			6.5120,6.4740,6.4295,6.3784,6.3206,6.2560,6.1845,6.1060,6.0204,5.9276,5.8275,5.7200,5.6050,\\
			5.4824,5.3521,5.2140,5.0680,4.9140,4.7519,4.5816,4.4030,4.2160,4.0205,3.8164,3.6036,3.3820,\\
			3.1515,2.9120,2.6634,2.4056,2.1385,1.8620,1.5760,1.2804,0.9751,0.6600,0.3350]^{\rm T}
		\end{multline*}
		
		\item 
		Use \verb|spdiags| once to form a sparse representation of the coefficient matrix and then use the backslash operator to solve the system.
		\lstinputlisting{hw2p4b.m}
		produces the same result above.
		
		\item 
		Use \verb|tridisolve| to solve the system.
		\lstinputlisting{hw2p4c.m}
		produces the same result above.
		
		\item 
		Use \verb|condest| to estimate the condition of the coefficient matrix.\lstinputlisting{hw2p4d.m}
		outputs condition number = 5100 using $L_1$ norm.
	\end{enumerate}
	
\end{enumerate}

\end{document}


