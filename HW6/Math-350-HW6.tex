\documentclass[10pt]{report}

\usepackage{verbatim}
\usepackage{subcaption} % for subfigures
%\usepackage{amsthm} % for QED
%\usepackage{algpseudocode} % for pseudo-code
\usepackage{mathtools} % for \xRightarrow

\usepackage{listings} % for code
\lstset
{
	language=Matlab,
	frame=single,
	basicstyle=\footnotesize,
	numbers=left,
	stepnumber=1,
	showstringspaces=false,
	tabsize=4,
	breaklines=true,
	breakatwhitespace=false,
}

\usepackage{siunitx} % for scientific notation
% for `e' in scientific notation
\sisetup{output-exponent-marker=\ensuremath{\mathrm{e}}}

\usepackage{float} % for figure [H]
\usepackage{booktabs} % for tabular
\usepackage{caption} % for \caption*
\usepackage[export]{adjustbox} % for valign=t
\usepackage{array} % for column type m
\usepackage{verbatim}
\usepackage{graphicx}
\graphicspath{ {imgs/} }
\usepackage{fancyhdr}
\usepackage{amssymb}
\usepackage{amsmath}

%%%%%% Pagination
\setlength{\topmargin}{-.3 in}
\setlength{\oddsidemargin}{0in}
\setlength{\evensidemargin}{0in}
\setlength{\textheight}{9.in}
\setlength{\textwidth}{6.5in}

%Title page
\newcommand{\hwTitle}{Homework \#6}
\newcommand{\hwCourse}{Introduction to Computational Mathematics}
\newcommand{\hmwkClassInstructor}{Professor Shuwang Li}

\title{
	\vspace{2in}
	\textmd{\textbf{\hwCourse\\\hwTitle}}\\
	\vspace{0.3in}\large{\textit{\hmwkClassInstructor}}
	\vspace{3in}
}

%\title{Homework 1}
\author{\textbf{Zhihao Ai}}
\date{}

%Header setting.
\pagestyle{fancy}
\fancyhead[L]{Zhihao Ai}
\fancyhead[C]{Math 350}
\fancyhead[R]{Homework 6}
%%%%%%

%Custom commands.
\newcommand{\ds}{\displaystyle}
\newcommand{\eva}[2] {\left. #1 \right|_{#2}}
\newcommand{\dintt}[4] {\int_{#1}^{#2} #3 d#4}

\newcolumntype{C}{ >{\centering\arraybackslash} m{3em} }
\newcolumntype{D}{ >{\centering\arraybackslash} m{4em} }
\newcolumntype{N}{ >$ c <$}

\newcommand{\abs}[1] {\left| #1 \right|}
\newcommand{\norm}[2][\infty] {\left\Vert \mathbf{#2} \right\Vert_#1}

\begin{document}

\maketitle

\section*{Part 1. Reading Assignment}
%Read chapter 6.

\section*{Part 2. Fundamental Concepts/Ideas}
\begin{enumerate}
	\item 
	Consider finding an approximation to integral: $I = \dintt{0}{\pi/2}{\cos(x/2)}{x}$ using composite trapezoidal rule, $I = \dintt{a}{b}{f(x)}{x} = \frac{h}{2}[f(a) + 2\sum_{k=1}^{n-1}f(a+kh) + f(b) ] + error$, where $error = -\frac{b-a}{12}h^2 f''(\eta)$ for some $\eta\in [a,b]$. Suppose we wish to obtain an approximation with the absolute value of the error term less than $0.0001$, determine the number of subintervals $n$ correspondingly. (\textit{\textbf{hint}}: $\cos(\eta) \le 1$)
	
	\item 
	Redo problem 1 using composite Simpson’s rule $I = \dintt{a}{b}{f(x)}{x} = \frac{h}{3}[f(a) + 4\sum_{k=1, \text{odd}}^{n/2-1}f(a+kh) $
	$ + 2\sum_{k=2, \text{even}}^{n/2}f(a+kh) + f(b) ] + error$ where $error = -\frac{b-a}{180}h^4f^{(4)}(\eta)$, for some $\eta\in [a,b]$. What can you learn from problem 1 and 2?
	
	\item 
	Review your lecture notes and derive the 2-point Gaussian-Legendre formula, $\dintt{-1}{1}{f(x)}{x} \approx w_1 f(x_1) + w_2 f(x_2)$, i.e. you need to determine the weights $w_1$ and $w_2$ and the nodes $x_1$ and $x_2$ so that the above rule is exact for the functions $f(x)=1, f(x)=x, f(x)=x^2,$ and $f(x)=x^3$. Evaluate integral $i = \dintt{1}{4}{\frac{1}{x+2018}}{x}$ using the formula you derived. What is the exact value? (\textit{\textbf{hint}}: you have to map the\\[0.5 ex]
	interval $[1, 4]$ to $[-1,1]$ following your lecture notes.)
	
	\item 
	In class, we derived a three-point centered difference formula to approximate $f'(x)$. Using Taylor expansion, derive the following new difference formulas: $f'(x)\approx a_0 f(x) + a_{-1}f(x-h) + a_{-2}f(x-2h)$, which is known as three-point backward differentiation formula. What is the error term?
\end{enumerate}

\section*{Part 3. Computer Assignments}
\begin{enumerate}
	\item 
	(Problem 6.6 @ Page 14)  The error function erf$(x)$ is defined by an integral:
	\[
	\text{erf}(x) = \frac{2}{\sqrt{\pi}} \dintt{0}{x}{e^{-x^2}}{x}
	\]
	Use \texttt{quadtx} to tabulate erf$(x)$ for $x = 0.1, 0.2, . . . , 1.0$. Compare the results with the built-in Matlab function \texttt{erf(x)}.
	
	\item 
	(Problem 6.9 @ Page 15)
	\begin{enumerate}
		\item [(a)]
		What is the exact value of
		\[
		\dintt{0}{4\pi}{\cos^2 x}{x}
		\]
		
		\item [(b)]
		What does \texttt{quadtx} compute for this integral? Why is it wrong?
		
		\item [(c)]
		How does \texttt{quad} overcome the difficulty?
	\end{enumerate}
\end{enumerate}
\end{document}


