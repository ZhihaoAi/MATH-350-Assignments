\documentclass[10pt]{report}

\usepackage{verbatim}
\usepackage{subcaption} % for subfigures
%\usepackage{amsthm} % for QED
%\usepackage{algpseudocode} % for pseudo-code
\usepackage{mathtools} % for \xRightarrow

\usepackage{listings} % for code
\lstset
{
	language=Matlab,
	frame=single,
	basicstyle=\footnotesize,
	numbers=left,
	stepnumber=1,
	showstringspaces=false,
	tabsize=4,
	breaklines=true,
	breakatwhitespace=false,
}

\usepackage{siunitx} % for scientific notation
% for `e' in scientific notation
\sisetup{output-exponent-marker=\ensuremath{\mathrm{e}}}

\usepackage{float} % for figure [H]
\usepackage{booktabs} % for tabular
\usepackage{caption} % for \caption*
\usepackage[export]{adjustbox} % for valign=t
\usepackage{array} % for column type m
\usepackage{verbatim}
\usepackage{graphicx}
\graphicspath{ {imgs/} }
\usepackage{fancyhdr}
\usepackage{amssymb}
\usepackage{amsmath}

%%%%%% Pagination
\setlength{\topmargin}{-.3 in}
\setlength{\oddsidemargin}{0in}
\setlength{\evensidemargin}{0in}
\setlength{\textheight}{9.in}
\setlength{\textwidth}{6.5in}

%Title page
\newcommand{\hwTitle}{Homework \#6}
\newcommand{\hwCourse}{Introduction to Computational Mathematics}
\newcommand{\hmwkClassInstructor}{Professor Shuwang Li}

\title{
	\vspace{2in}
	\textmd{\textbf{\hwCourse\\\hwTitle}}\\
	\vspace{0.3in}\large{\textit{\hmwkClassInstructor}}
	\vspace{3in}
}

%\title{Homework 1}
\author{\textbf{Zhihao Ai}}
\date{}

%Header setting.
\pagestyle{fancy}
\fancyhead[L]{Zhihao Ai}
\fancyhead[C]{Math 350}
\fancyhead[R]{Homework 6}
%%%%%%

%Custom commands.
\newcommand{\ds}{\displaystyle}
\newcommand{\eva}[2] {\left. #1 \right|_{#2}}
\newcommand{\dintt}[4] {\int_{#1}^{#2} #3 d#4}

\newcolumntype{C}{ >{\centering\arraybackslash} m{3em} }
\newcolumntype{D}{ >{\centering\arraybackslash} m{4em} }
\newcolumntype{N}{ >$ c <$}

\DeclarePairedDelimiter\autoparen{(}{)}
\newcommand{\pa}[1]{\autoparen*{#1}}
\newcommand{\abs}[1] {\left| #1 \right|}
\newcommand{\norm}[2][\infty] {\left\Vert \mathbf{#2} \right\Vert_#1}

\begin{document}

\maketitle

\section*{Part 1. Reading Assignment}
%Read chapter 6.

\section*{Part 2. Fundamental Concepts/Ideas}
\begin{enumerate}
	\item 
	Consider finding an approximation to integral: $I = \dintt{0}{\pi/2}{\cos(x/2)}{x}$ using composite trapezoidal rule, $I = \dintt{a}{b}{f(x)}{x} = \frac{h}{2}[f(a) + 2\sum_{k=1}^{n-1}f(a+kh) + f(b) ] + error$, where $error = -\frac{b-a}{12}h^2 f''(\eta)$ for some $\eta\in [a,b]$. Suppose we wish to obtain an approximation with the absolute value of the error term less than $0.0001$, determine the number of subintervals $n$ correspondingly. (\textit{\textbf{hint}}: $\cos(\eta) \le 1$)
	\begin{align*}
		\abs{error} = \abs{-\frac{b-a}{12}h^2 \cos''(\eta)}
		= \abs{-\frac{\pi/2}{12} \pa{\frac{\pi/2}{n}}^2 \pa{-\cos(\eta)}}
		\le \frac{\pi^3}{96n^2}
		\le 0.0001
		\to
		n \ge 57
	\end{align*}
	Therefore the number of subintervals should be at least 57.
	
	\item 
	Redo problem 1 using composite Simpson’s rule $I = \dintt{a}{b}{f(x)}{x} = \frac{h}{3}[f(a) + 4\sum_{k=1, \text{odd}}^{n/2-1}f(a+kh) $
	$ + 2\sum_{k=2, \text{even}}^{n/2}f(a+kh) + f(b) ] + error$ where $error = -\frac{b-a}{180}h^4f^{(4)}(\eta)$, for some $\eta\in [a,b]$. What can you learn from problem 1 and 2?
	\begin{align*}
	\abs{error} = \abs{-\frac{b-a}{180}h^4f^{(4)}(\eta)}
	= \abs{-\frac{\pi/2}{180} \pa{\frac{\pi/2}{n}}^4 \cos(\eta)}
	\le \frac{\pi^5}{5760n^4}
	\le 0.0001
	\to
	n \ge 5
	\end{align*}
	Therefore the number of subintervals should be at least 5.
	
	\textit{Things learned:} Composite trapezoidal rule is of order 2 and composite Simpson's rule is of order 4 so the composite Simpson's rule converges to the real value faster. Therefore, to achieve the same level of accuracy, the number of subintervals needed for composite trapezoidal rule is much larger than that for composite Simpson's rule.
	
	\item 
	Review your lecture notes and derive the 2-point Gaussian-Legendre formula, $\dintt{-1}{1}{f(x)}{x} \approx w_1 f(x_1) + w_2 f(x_2)$, i.e. you need to determine the weights $w_1$ and $w_2$ and the nodes $x_1$ and $x_2$ so that the above rule is exact for the functions $f(x)=1, f(x)=x, f(x)=x^2,$ and $f(x)=x^3$. Evaluate integral $I = \dintt{1}{4}{\frac{1}{x+2018}}{x}$ using the formula you derived. What is the exact value? (\textit{\textbf{hint}}: you have to map the\\[0.5 ex]
	interval $[1, 4]$ to $[-1,1]$ following your lecture notes.)
	\[
	\begin{dcases}
	2 = \dintt{-1}{1}{1}{x} = w_1\cdot 1 + w_2\cdot 1\\
	0 = \dintt{-1}{1}{x}{x} = w_1\cdot x_1 + w_2\cdot x_2\\
	\frac{2}{3} = \dintt{-1}{1}{x^2}{x} = w_1\cdot x_1^2 + w_2\cdot x_2^2\\
	0 = \dintt{-1}{1}{x^3}{x} = w_1\cdot x_1^3 + w_2\cdot x_2^3
	\end{dcases}
	\Rightarrow
	\begin{dcases}
	w_1 = 1\\
	w_2 = 1\\
	x_1 = -\frac{1}{\sqrt{3}}\\
	x_2 = \frac{1}{\sqrt{3}}\\
	\end{dcases}
	\]
	$\dintt{1}{4}{\frac{1}{x+2018}}{x} = \ln(\frac{674}{673}) \approx 0.00148478$. To use the 2-point Gaussian-Legendre formula, let $x = \frac{3}{2}t+\frac{5}{2}, dx = \frac{3}{2}dt, t\in [-1, 1]$. Then 
	\begin{align*}
	\dintt{1}{4}{\frac{1}{x+2018}}{x} 
	&= \dintt{-1}{1}{f\left(\frac{3}{2}t+\frac{5}{2}\right)\cdot\frac{3}{2}}{t} \\
	&\approx \frac{3}{2} \left[
	f\left(\frac{3}{2} \pa{-\frac{1}{\sqrt{3}}} + \frac{5}{2}\right) 
	+ f\left(\frac{3}{2} \pa{\frac{1}{\sqrt{3}}} + \frac{5}{2}\right) 
	\right]\\
	&\approx 0.00148478
	\end{align*}
	
	\item 
	In class, we derived a three-point centered difference formula to approximate $f'(x)$. Using Taylor expansion, derive the following new difference formulas: $f'(x)\approx a_0 f(x) + a_{-1}f(x-h) + a_{-2}f(x-2h)$, which is known as three-point backward differentiation formula. What is the error term?
	
	Apply Taylor expansion on the RHS,
	\begin{align*}
		&\phantom{\ =\ } a_0 f(x) + a_{-1}f(x-h) + a_{-2}f(x-2h)\\
		&= a_0 f(x) + a_{-1} \pa{f(x) - hf'(x) + \frac{h^2}{2} f''(x) + O(h^3)} + a_{-2} \pa{f(x) - 2hf'(x) + 2h^2 f''(x) + O(h^3)}\\
		&= (a_0 + a_1 + a_2)f(x) - h(a_{-1} + 2a_{-2})f'(x) + h^2\pa{\frac{a_{-1}}{2} + 2a_{-2}} + O(h^3)\\
		&= f'(x)
	\end{align*}
	Solving the equations, we have $a_0 = \frac{3}{2h}, a_{-1} = -\frac{2}{h}, a_{-2} = \frac{1}{2h}$. Therefore, the formula is
	\[
	f'(x) = \frac{1}{h}\pa{\frac{3}{2} f(x) - 2f(x-h) + \frac{1}{2}f(x-2h)}
	\]
	The error term is $O(h^3)$, or more exactly, $-h^3\pa{\frac{1}{6}f'''(\xi_1) + \frac{4}{3}f'''(\xi_2)}$.
	
\end{enumerate}

\section*{Part 3. Computer Assignments}
\begin{enumerate}
	\item 
	(Problem 6.6 @ Page 14)  The error function erf$(x)$ is defined by an integral:
	\[
	\text{erf}(x) = \frac{2}{\sqrt{\pi}} \dintt{0}{x}{e^{-x^2}}{x}
	\]
	Use \texttt{quadtx} to tabulate erf$(x)$ for $x = 0.1, 0.2, . . . , 1.0$. Compare the results with the built-in Matlab function \texttt{erf(x)}.
	
	\lstinputlisting{hw6ca1.m}
	\begin{table}[H]
		\centering
		\begin{tabular}{*{3}{N}} 
			\toprule
			x & \texttt{erf(x)} & erf(x) \text{ by } \texttt{quadtx} \\ \midrule
			0.1 & 0.1124629160 & 0.1124629160\\
			0.2 & 0.2227025892 & 0.2227025892\\
			0.3 & 0.3286267595 & 0.3286267593\\
			0.4 & 0.4283923550 & 0.4283923546\\
			0.5 & 0.5204998778 & 0.5204998777\\
			0.6 & 0.6038560908 & 0.6038560907\\
			0.7 & 0.6778011938 & 0.6778012053\\
			0.8 & 0.7421009647 & 0.7421009650\\
			0.9 & 0.7969082124 & 0.7969082132\\
			1.0 & 0.8427007929 & 0.8427007951\\
			\bottomrule
		\end{tabular}
	\end{table}
	
	\item 
	(Problem 6.9 @ Page 15)
	\begin{enumerate}
		\item [(a)]
		What is the exact value of
		\[
		\dintt{0}{4\pi}{\cos^2 x}{x}
		\]
		The exact value is $2\pi$, which is approximately 6.28319.
		
		\item [(b)]
		What does \texttt{quadtx} compute for this integral? Why is it wrong?
		
		\texttt{quadtx} produces 12.566371. The first time it applys Simpson's rule, it chooses the points $(\pi, 1)$, $(2\pi, 1)$ and $(3\pi, 1)$, which produces 12.566371; when it applys Simpson's rule for the two subintervals, the points still have the same $y$-value 1. Because the difference between them is 0, it stops there and gives the wrong answer.
		
		\item [(c)]
		How does \texttt{quad} overcome the difficulty?
		
		\texttt{quad} uses the extrapolated Simpson’s rule in an adaptive recursive algorithm whereas \texttt{quadtx} evaluates the integrand three times to
		give the first, unextrapolated, Simpson’s rule estimate.
	\end{enumerate}
\end{enumerate}
\end{document}


