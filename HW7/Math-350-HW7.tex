\documentclass[10pt]{report}

\usepackage{verbatim}
\usepackage{subcaption} % for subfigures
%\usepackage{amsthm} % for QED
%\usepackage{algpseudocode} % for pseudo-code
\usepackage{mathtools} % for \xRightarrow

\usepackage{listings} % for code
\lstset
{
	language=Matlab,
	frame=single,
	basicstyle=\footnotesize,
	numbers=left,
	stepnumber=1,
	showstringspaces=false,
	tabsize=4,
	breaklines=true,
	breakatwhitespace=false,
}

\usepackage{siunitx} % for scientific notation
% for `e' in scientific notation
\sisetup{output-exponent-marker=\ensuremath{\mathrm{e}}}

\usepackage{float} % for figure [H]
\usepackage{booktabs} % for tabular
\usepackage{caption} % for \caption*
\usepackage[export]{adjustbox} % for valign=t
\usepackage{array} % for column type m
\usepackage{verbatim}
\usepackage{graphicx}
\graphicspath{ {imgs/} }
\usepackage{fancyhdr}
\usepackage{amssymb}
\usepackage{amsmath}

%%%%%% Pagination
\setlength{\topmargin}{-.3 in}
\setlength{\oddsidemargin}{0in}
\setlength{\evensidemargin}{0in}
\setlength{\textheight}{9.in}
\setlength{\textwidth}{6.5in}

%Title page
\newcommand{\hwTitle}{Homework \#7}
\newcommand{\hwCourse}{Introduction to Computational Mathematics}
\newcommand{\hmwkClassInstructor}{Professor Shuwang Li}

\title{
	\vspace{2in}
	\textmd{\textbf{\hwCourse\\\hwTitle}}\\
	\vspace{0.3in}\large{\textit{\hmwkClassInstructor}}
	\vspace{3in}
}

%\title{Homework 1}
\author{\textbf{Zhihao Ai}}
\date{}

%Header setting.
\pagestyle{fancy}
\fancyhead[L]{Zhihao Ai}
\fancyhead[C]{Math 350}
\fancyhead[R]{Homework 7}
%%%%%%

%Custom commands.
\newcommand{\ds}{\displaystyle}
\newcommand{\eva}[2] {\left. #1 \right|_{#2}}
\newcommand{\dintt}[4] {\int_{#1}^{#2} #3 d#4}

\newcolumntype{C}{ >{\centering\arraybackslash} m{3em} }
\newcolumntype{D}{ >{\centering\arraybackslash} m{4em} }
\newcolumntype{N}{ >$ c <$}

\DeclarePairedDelimiter\autoparen{(}{)}
\newcommand{\pa}[1]{\autoparen*{#1}}
\newcommand{\abs}[1] {\left| #1 \right|}
\newcommand{\norm}[2][\infty] {\left\Vert \mathbf{#2} \right\Vert_#1}

\begin{document}

\maketitle

\section*{Part 1. Reading Assignment}
%Read chapter 7.

\section*{Part 2. Fundamental Concepts/Ideas}
\begin{enumerate}
	\item 
	In class, we derived the centered difference formula using Taylor expansions for
	\begin{align*}
		f(x+h) &= f(x) + hf'(x) + \frac{h^2}{2}f''(x) + \frac{h^3}{6}f'''(\xi_1)\\
		f(x-h) &= f(x) - hf'(x) + \frac{h^2}{2}f''(x) - \frac{h^3}{6}f'''(\xi_2)
	\end{align*}
	\begin{enumerate}
		\item 
		Show that $\ds f''(x) \approx \frac{f(x+h) - 2f(x) + f(x-h)}{h^2}$.
		
		Add the equations together and rearrange them, we have
		\begin{align*}
			f(x+h) + f(x-h) &= 2f(x) + h^2 f''(x) + \frac{h^3}{6}f'''(\xi_1) - \frac{h^3}{6}f'''(\xi_2)\\
			h^2 f''(x) &= f(x+h) - 2f(x) + f(x-h) +\frac{h^3}{6}f'''(\xi_2) - \frac{h^3}{6}f'''(\xi_1)\\
			f''(x) &\approx \frac{f(x+h) - 2f(x) + f(x-h)}{h^2}
		\end{align*}
		
		\item
		Consider the quadratic interpolant $p(x)$ passing through three points, $(x_1, f(x_1))$, $(x_2, f(x_2))$, and $(x_3, f(x_3))$. In class, we derived the centered
		difference formula based on an approximation $f'(x) \approx p'(x)$. Show that the difference formula you derived in part (a) can be obtained using approximation $f''(x) \approx p''(x)$, with $x_1 = x - h$, $x_2 = x$, and $x_3 = x + h$.
		\begin{align*}
			p(x) &= \frac{(x-x_2)(x-x_3)}{(x_1 - x_2)(x_1 - x_3)}f(x_1) 
			+ \frac{(x-x_1)(x-x_3)}{(x_2 - x_1)(x_2 - x_3)}f(x_2)
			+ \frac{(x-x_1)(x-x_2)}{(x_3 - x_1)(x_3 - x_2)}f(x_3)\\
			p''(x) &= \frac{1}{h^2} f(x_1) - \frac{2}{h^2} f(x_2) + \frac{1}{h^2} f(x_3)\\
			p''(x) &= \frac{f(x+h) - 2f(x) + f(x-h)}{h^2}
 		\end{align*}
		Therefore, $f''(x) \approx p''(x)$.
		
	\end{enumerate}

	\item 
	Convert the following higher order IVPs into first order systems.
	\begin{enumerate}
		\item 
		$\ds m\frac{d^2y}{dt^2} + \gamma \frac{dy}{dt} + \frac{mg}{l} \sin(y) = A\sin(\Omega t)$ with $y(0) = \theta_0$, $y'(0) = \omega_0$.
		
		Let $u_1(t) = y(t), u_2(t) = y'(t)$, then we have
		\[
		\left\{
		\begin{aligned}
			u_1'(t) &= u_2(t)\\
			u_2'(t) &= -\frac{\gamma}{m} u_2(t) - \frac{g}{l}\sin(u_1(t)) + \frac{A}{m}\sin(\omega t)\\
			u_1(0) &= \theta_0\\
			u_2(0) &= \omega_0
		\end{aligned}
		\right.
		\]
		
		\item
		$\ds \frac{d^3y}{dt^3} + \pa{\frac{d^2y}{dt^2}}^2 - 3\pa{\frac{dy}{dt}}^3 + \cos^2(y) = e^{-t}\sin(3t)$, with $y(1) = 1, y'(1) = 2, y''(1) = 0$.
		
		Let $u_1(t) = y(t), u_2(t) = y'(t), u_3(t) = y''(t)$, then we have
		\[
		\left\{
		\begin{aligned}
		u_1'(t) &= u_2(t)\\
		u_2'(t) &= u_3(t)\\
		u_3'(t) &= -u_3^2(t) + 3u_2^2(t) - \cos^2(u_1(t)) + e^{-t}\sin(3t)\\
		u_1(1) &= 1\\
		u_2(1) &= 2\\
		u_3(1) &= 0\\
		\end{aligned}
		\right.
		\]
	\end{enumerate}
\end{enumerate}

\section*{Part 3. Computer Assignments}
\begin{enumerate}
	\item 
	(Problem 7.4 @ Page 35) The error function erf($x$) is usually defined by an integral,
	\[
	\text{erf}(x) = \frac{2}{\sqrt{\pi}} \dintt{0}{x}{e^{-x^2}}{x},
	\]
	but it can also be defined as the solution to the differential equation
	\begin{align*}
		y'(x) &= \frac{2}{\sqrt{\pi}}e^{-x^2},\\
		y(0) &= 0.
	\end{align*}
	Use \texttt{ode23tx} to solve this differential equation on the interval $0 \le x \le 2$. Compare the results with the built-in Matlab function \texttt{erf(x)} at the points chosen by \texttt{ode23tx}.
	
	\lstinputlisting{hw7ca1.m}
	Using \texttt{ode23tx}, $\text{erf}(2) \approx 0.995271$. It is close to what \texttt{erf(2)} outputs, which is $0.995322$.
	
	\item 
	Consider the following IVP,
	\[
	\frac{dy}{dt} = -2y^2 + \frac{1}{1+t^2},\quad \text{with } y(0) = 0,
	\]
	\begin{enumerate}
		\item 
		Show that the exact solution is $y(t) = \frac{t}{1+t^2}$.
		\[
		-2y^2 + \frac{1}{1+t^2} = -2\cdot\frac{t^2}{(1+t^2)^2} + \frac{1}{1+t^2} = \frac{1 - t^2}{(1+t^2)^2} = \frac{dy}{dt}
		\]
		
		\item 
		Solve this IVP numerically for $0 \le t \le 0.3$ and calculate the numerical errors at time $t=0.1, 0.2$ and $0.3$
		\begin{enumerate}
			\item 
			using explicit Euler’s method with $h=0.025$.
			
			\item 
			using classical $2^{\text{nd}}$ order Runge-Kutta method with $h=0.05$.
			
			\item 
			using $4^{\text{th}}$ order Runge-Kutta method with $h=0.1$.
		\end{enumerate}
	
		\lstinputlisting{hw7ca2.m}
		\begin{table}[H]
			\centering
			\begin{tabular}{*{4}{N}} 
				\toprule
				 & \text{Explicit Euler} & \text{ERK2} & \text{ERK4} \\ \midrule
				y(0.3) & 0.277879 & 0.274868 & 0.275228 \\
				\epsilon_{0.1} & 0.000337 & 0.000126 & 0.000000 \\
				\epsilon_{0.2} & 0.001310 & 0.000249 & 0.000001 \\
				\epsilon_{0.3} & 0.002650 & 0.000361 & 0.000002 \\
				\bottomrule
			\end{tabular}
		\end{table}
	\end{enumerate}
	
	\item 
	Redo the above problem 2 using a two-step Adams-Bashforth method
	$0 \le t \le 0.3$. You can use the explicit Euler's method to generate the necessary initial values, let $h=0.025$.
	\lstinputlisting{hw7ca3.m}
	AB2 produces $y(0.3)\approx 0.275533$. $\epsilon_{0.1} = 0.000128, \epsilon_{0.2} = 0.000246, \epsilon_{0.3} = 0.000303$.
\end{enumerate}

\end{document}


