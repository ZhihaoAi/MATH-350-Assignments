\documentclass[10pt]{report}

\usepackage{verbatim}
\usepackage{subcaption} % for subfigures
%\usepackage{amsthm} % for QED
%\usepackage{algpseudocode} % for pseudo-code
\usepackage{mathtools} % for \xRightarrow

\usepackage{listings} % for code
\lstset
{
	language=Matlab,
	frame=single,
	basicstyle=\footnotesize,
	numbers=left,
	stepnumber=1,
	showstringspaces=false,
	tabsize=4,
	breaklines=true,
	breakatwhitespace=false,
}

\usepackage{siunitx} % for scientific notation
% for `e' in scientific notation
\sisetup{output-exponent-marker=\ensuremath{\mathrm{e}}}

\usepackage{float} % for figure [H]
\usepackage{booktabs} % for tabular
\usepackage{caption} % for \caption*
\usepackage[export]{adjustbox} % for valign=t
\usepackage{array} % for column type m
\usepackage{verbatim}
\usepackage{graphicx}
\graphicspath{ {imgs/} }
\usepackage{fancyhdr}
\usepackage{amssymb}
\usepackage{amsmath}

%%%%%% Pagination
\setlength{\topmargin}{-.3 in}
\setlength{\oddsidemargin}{0in}
\setlength{\evensidemargin}{0in}
\setlength{\textheight}{9.in}
\setlength{\textwidth}{6.5in}

%Title page
\newcommand{\hwTitle}{Homework \#5}
\newcommand{\hwCourse}{Introduction to Computational Mathematics}
\newcommand{\hmwkClassInstructor}{Professor Shuwang Li}

\title{
	\vspace{2in}
	\textmd{\textbf{\hwCourse\\\hwTitle}}\\
	\vspace{0.3in}\large{\textit{\hmwkClassInstructor}}
	\vspace{3in}
}

%\title{Homework 1}
\author{\textbf{Zhihao Ai}}
\date{}

%Header setting.
\pagestyle{fancy}
\fancyhead[L]{Zhihao Ai}
\fancyhead[C]{Math 350}
\fancyhead[R]{Homework 5}
%%%%%%

%Custom commands.
\newcommand{\ds}{\displaystyle}
\newcommand{\eva}[2] {\left. #1 \right|_{#2}}
\newcommand{\dintt}[4] {\int_{#1}^{#2} #3 d#4}

\newcolumntype{C}{ >{\centering\arraybackslash} m{3em} }
\newcolumntype{D}{ >{\centering\arraybackslash} m{4em} }
\newcolumntype{N}{ >$ c <$}

\newcommand{\abs}[1] {\left| #1 \right|}
\newcommand{\norm}[2][\infty] {\left\Vert \mathbf{#2} \right\Vert_#1}

\begin{document}

\maketitle

\section*{Part 1. Reading Assignment}
%Read chapter 5.

\section*{Part 2. Fundamental Concepts/Ideas}
\begin{enumerate}
	\item 
	Given data
	\begin{table}[H]
		\centering
		\begin{tabular}{*{6}{N}} \toprule
			i & 1 & 2 & 3 & 4 & 5 \\ \midrule
			x_i & 1.0 & 1.4 & 1.8 & 2.2 & 2.6\\
			y_i & 0.931 & 0.473 & 0.297 & 0.224 & 0.618\\
			\bottomrule
		\end{tabular}
	\end{table}
	We can fit these data using a curve $y=p(x)=\frac{1}{a+bx}$ in the least square sense.
	\begin{enumerate}
		\item 
		Find coefficients $a$ and $b$. Hint: you can let $Y(x)=1/p(x)=a+bx$.
		
		\item 
		Compute the residual $E = \sum_{i=1}^{5} [p(x_i) - y_i]^2$.
		
		\item 
		Use Matlab to plot the curve $y=p(x)$ and $(x_i, y_i)$ on the same plot.
	\end{enumerate}

	\item 
	Given data
	\begin{table}[H]
		\centering
		\begin{tabular}{*{8}{N}} \toprule
			i & 1 & 2 & 3 & 4 & 5 & 6 & 7 \\ \midrule
			x_i & 0.2 & 0.3 & 0.4 & 0.5 & 0.6 & 0.7 & 0.8\\
			y_i & 3.16 & 2.38 & 1.75 & 1.34 & 1.00 & 0.74 & 0.56\\
			\bottomrule
		\end{tabular}
	\end{table}
	We can fit these data using a curve $y=q(x)=\beta e^{-\alpha x}$ in the least square sense.
	\begin{enumerate}
		\item 
		Find coefficients $\alpha$ and $\beta$. Hint: you can let $Y(x)=\ln y=\ln \beta - \alpha x$.
		
		\item 
		Compute the residual $E = \sum_{i=1}^{7} [q(x_i) - y_i]^2$.
		
		\item 
		Use Matlab to plot the curve $y=q(x)$ and $(x_i, y_i)$ on the same plot.
	\end{enumerate}
\end{enumerate}

\section*{Part 3. Computer assignments}
(Problem 5.8 @ Page 22) Given 25 observations, $y_k$, taken at equally spaced values of $t$.
\begin{enumerate}
	\item 
	Fit the data with a straight line, $y(t) = \beta_1 + \beta_2t$, and plot the residuals, $y(t_k)−y_k$. You should observe that one of the data points has a much larger residual than the others. This is probably an \textit{outlier}.
	
	\item 
	Discard the outlier, and fit the data again by a straight line. Plot the residuals again. Do you see any pattern in the residuals?
	
	\item 
	Fit the data, with the outlier excluded, by a model of the form
	\[
	y(t) = \beta_1 + \beta_2 t + \beta_3 \sin t
	\]
	
	\item 
	Evaluate the third fit on a finer grid over the interval $[0,26]$. Plot the fitted curve, using line style '-', together with the data, using line style 'o'. Include the outlier, using a different marker, '*'.
\end{enumerate}
\end{document}


