\documentclass[10pt]{report}

\usepackage{verbatim}
\usepackage{subcaption} % for subfigures
%\usepackage{amsthm} % for QED
%\usepackage{algpseudocode} % for pseudo-code
\usepackage{mathtools} % for \xRightarrow

\usepackage{listings} % for code
\lstset
{
	language=Matlab,
	frame=single,
	basicstyle=\footnotesize,
	numbers=left,
	stepnumber=1,
	showstringspaces=false,
	tabsize=4,
	breaklines=true,
	breakatwhitespace=false,
}

\usepackage{siunitx} % for scientific notation
% for `e' in scientific notation
\sisetup{output-exponent-marker=\ensuremath{\mathrm{e}}}

\usepackage{float} % for figure [H]
\usepackage{booktabs} % for tabular
\usepackage{caption} % for \caption*
\usepackage[export]{adjustbox} % for valign=t
\usepackage{array} % for column type m
\usepackage{verbatim}
\usepackage{graphicx}
\graphicspath{ {imgs/} }
\usepackage{fancyhdr}
\usepackage{amssymb}
\usepackage{amsmath}

%%%%%% Pagination
\setlength{\topmargin}{-.3 in}
\setlength{\oddsidemargin}{0in}
\setlength{\evensidemargin}{0in}
\setlength{\textheight}{9.in}
\setlength{\textwidth}{6.5in}

%Title page
\newcommand{\hwTitle}{Homework \#4}
\newcommand{\hwCourse}{Introduction to Computational Mathematics}
\newcommand{\hmwkClassInstructor}{Professor Shuwang Li}

\title{
	\vspace{2in}
	\textmd{\textbf{\hwCourse\\\hwTitle}}\\
	\vspace{0.3in}\large{\textit{\hmwkClassInstructor}}
	\vspace{3in}
}

\author{\textbf{Zhihao Ai}}
\date{}

%Header setting.
\pagestyle{fancy}
\fancyhead[L]{Zhihao Ai}
\fancyhead[C]{Math 350}
\fancyhead[R]{Homework 4}
%%%%%%

%Custom commands.
\newcommand{\ds}{\displaystyle}
\newcommand{\eva}[2] {\left. #1 \right|_{#2}}
\newcommand{\dintt}[4] {\int_{#1}^{#2} #3 d#4}

\newcolumntype{C}{ >{\centering\arraybackslash} m{3em} }
\newcolumntype{D}{ >{\centering\arraybackslash} m{4em} }
\newcolumntype{N}{ >$ c <$}

\newcommand{\abs}[1] {\left| #1 \right|}
\newcommand{\norm}[2][\infty] {\left\Vert \mathbf{#2} \right\Vert_#1}

\begin{document}

\maketitle

\section*{Part 1. Reading Assignment}
%Read section 4.7,4.8, 4.9, and 4.10

\section*{Part 2. Fundamental Concepts/Ideas}
\begin{enumerate}
	\item 
	Consider nonlinear equation $\ln x + x - 1.5 = 0, x\in [1,2]$.
	\begin{enumerate}
		\item 
		Show that there exists a root in the interval using Intermediate Value Theorem.
		
		Since $f(x) = \ln x + x - 1.5$ is continuous and $f(1) < 0 < f(2)$, by intermediate value theorem, there exists a $c\in (1,2)$ where $f(c) = 0$.
		
		\item 
		Find the root using bi-section method, just do 6 iterations.
		\begin{table}[H]
			\centering
			\begin{tabular}{*{7}{N}} \toprule
				k & a & f(a) & b & f(b) & x^{(k)} & f(x) \\ \midrule
				1 & 1.000000 & -0.500000 & 2.000000 & 1.193147 & 1.500000 & 0.405465\\
				2 & 1.000000 & -0.500000 & 1.500000 & 0.405465 & 1.250000 & -0.026856\\
				3 & 1.250000 & -0.026856 & 1.500000 & 0.405465 & 1.375000 & 0.193454\\
				4 & 1.250000 & -0.026856 & 1.375000 & 0.193454 & 1.312500 & 0.084434\\
				5 & 1.250000 & -0.026856 & 1.312500 & 0.084434 & 1.281250 & 0.029086\\
				6 & 1.250000 & -0.026856 & 1.281250 & 0.029086 & 1.265625 & 0.001191\\
				\bottomrule
			\end{tabular}
		\end{table}
		After 6 iterations, the approximated root is $x\approx 1.265625$.
	\end{enumerate}

	\item 
	Solve nonlinear equation $x^6 - x - 1 = 0, x\in [1,2]$. The root is $x=1.134724138$. You may start with $x=1.5$.
	\begin{enumerate}
		\item 
		using bi-section method with tolerance $\abs{x^{(k+1)} - x^{(k)}} \le 10^{-3}$
		
		\lstinputlisting{hw4p2a.m}
		\begin{table}[H]
			\centering
			\begin{tabular}{*{8}{N}} \toprule
				k & a & f(a) & b & f(b) & x^{(k)} & f(x) & tol\\ \midrule
				1 & 1.000000 & -1.000000 & 2.000000 & 61.000000 & 1.500000 & 8.890625 & 0.500000\\
				2 & 1.000000 & -1.000000 & 1.500000 & 8.890625 & 1.250000 & 1.564697 & 0.250000\\
				3 & 1.000000 & -1.000000 & 1.250000 & 1.564697 & 1.125000 & -0.097713 & 0.125000\\
				4 & 1.125000 & -0.097713 & 1.250000 & 1.564697 & 1.187500 & 0.616653 & 0.062500\\
				5 & 1.125000 & -0.097713 & 1.187500 & 0.616653 & 1.156250 & 0.233269 & 0.031250\\
				6 & 1.125000 & -0.097713 & 1.156250 & 0.233269 & 1.140625 & 0.061578 & 0.015625\\
				7 & 1.125000 & -0.097713 & 1.140625 & 0.061578 & 1.132812 & -0.019576 & 0.007812\\
				8 & 1.132812 & -0.019576 & 1.140625 & 0.061578 & 1.136719 & 0.020619 & 0.003906\\
				9 & 1.132812 & -0.019576 & 1.136719 & 0.020619 & 1.134766 & 0.000427 & 0.001953\\
				10 & 1.132812 & -0.019576 & 1.134766 & 0.000427 & 1.133789 & -0.009598 & 0.000977\\
				\bottomrule
			\end{tabular}
		\end{table}
		After 10 iterations, $tol = \abs{x^{(k+1)} - x^{(k)}} \le 10^{-3}$ and the approximated root is $x\approx 1.133789$.

		\item 
		using Newton's method with tolerance $\abs{x^{(k+1)} - x^{(k)}} \le 10^{-9}$
		
		\lstinputlisting{hw4p2b.m}
		\begin{table}[H]
			\centering
			\begin{tabular}{*{4}{N}} \toprule
				k & x^{(k)} & f(x) & tol\\ \midrule
				1 & 1.3004908836 & 2.5372641434 & 0.1995091164\\
				2 & 1.1814804164 & 0.5384585843 & 0.1190104672\\
				3 & 1.1394555903 & 0.0492352510 & 0.0420248261\\
				4 & 1.1347776252 & 0.0005503239 & 0.0046779650\\
				5 & 1.1347241453 & 0.0000000711 & 0.0000534799\\
				6 & 1.1347241384 & 0.0000000000 & 0.0000000069\\
				7 & 1.1347241384 & -0.0000000000 & 0.0000000000\\
				\bottomrule
			\end{tabular}
		\end{table}
		After 7 iterations, $tol = \abs{x^{(k+1)} - x^{(k)}} \le 10^{-9}$ and the approximated root is $x\approx 1.1347241384$.
	\end{enumerate}
\end{enumerate}

\section*{Part 3. Computer Assignments}
(Problem 4.15 @ Page )
Kepler's model of planetary orbits includes a quantity $E$, the eccentricity anomaly, that satisfies the equation
\[
M = E - e\sin E
\]
where $M$ is the mean anomaly and $e$ is the eccentricity of the orbit. For this exercise, take $M = 24.851090$ and $e = 0.1$.
\begin{enumerate}
	\item [(a)]
	Use \texttt{fzerotx} to solve for $E$. You can assign the appropriate values to \texttt{M} and \texttt{e} and then use them in the definition of a function of \texttt{E}.
	
	Let $F(E) = E - e\sin E - M$, by intermediate value theorem, since $F(0) < 0 < F(50)$, there exists a root in $(0,50)$.
	\lstinputlisting{hw4ca1a.m}
	The script above produces $E\approx 24.8203571825429065711432485841214656829833984375000000$.
	
	\item [(b)]
	An ``exact" formula for $E$ is known:
	\[
	E = M + 2\sum_{m=1}^{\infty} \frac{1}{m}J_m(me)\sin(mM),
	\]
	where $J_m(x)$ is the Bessel function of the first kind of order $m$. Use this formula, and \texttt{besselj(m,x)} in Matlab, to compute $E$. How many terms are needed? How does this value of $E$ compare to the value obtained with \texttt{fzerotx}?
	
	\lstinputlisting{hw4ca1b.m}
	The script above produces $E\approx 24.8203571825429030184295697836205363273620605468750000$ using 14 terms. This value is less accurate than the one obtained with \texttt{fzerotx}.
\end{enumerate}

\end{document}


